\documentclass{beamer}

\usetheme{CambridgeUS}
\usecolortheme{default}

\title{Implementación de un motor de búsqueda utilizando TF-IDF}
\author{Daniel Collazo Aldana}
\institute{UH facultad de Matematicas}
\date{\today}

\begin{document}

\begin{frame}
\titlepage
\end{frame}

\begin{frame}
\frametitle{Introducción}
En esta presentación se describirá la implementación de un motor de búsqueda utilizando el modelo TF-IDF. Se explorará cómo funciona el modelo y cómo se utilizó para calcular la relevancia de los documentos en función de una consulta de búsqueda.
\end{frame}

\begin{frame}
\frametitle{Modelo TF-IDF}
El modelo TF-IDF se utiliza para calcular la relevancia de un término en un documento en función de su frecuencia en el documento y su frecuencia en el corpus. El modelo se utiliza para calcular el vector de características de cada documento en función de la frecuencia de los términos en el documento y su importancia en el corpus.
\end{frame}

\begin{frame}
\frametitle{Implementación}
Se implementó un motor de búsqueda utilizando el modelo TF-IDF en Python. El programa tiene varias funciones estáticas que llevan a cabo diferentes tareas, como leer archivos de texto, normalizar documentos y calcular vectores TF-IDF. Estos vectores se utilizan para calcular la similitud coseno entre la consulta de búsqueda y los documentos. Finalmente, los resultados se ordenan por similitud descendente y se devuelven los 10 documentos más relevantes.
\end{frame}

\begin{frame}
\frametitle{Demo}
A continuación, se mostrará una demostración del motor de búsqueda implementado. Se realizará una consulta de búsqueda y se mostrarán los resultados más relevantes.
\end{frame}

\begin{frame}
\frametitle{Conclusiones}
En conclusión, se implementó un motor de búsqueda utilizando el modelo TF-IDF para calcular la relevancia de los documentos en función de una consulta de búsqueda. El modelo se utilizó para calcular los vectores de características de los documentos y se utilizó la similitud coseno para calcular la relevancia de los documentos. El motor de búsqueda es capaz de devolver los documentos más relevantes en función de la consulta de búsqueda. 
\end{frame}

\end{document} 