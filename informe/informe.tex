\documentclass{beamer}
\usetheme{Madrid}
\title{Informe del Moogle}
\author{Daniel Collazo Aldana}
\institute{Grupo C112}
\date{\today}

\begin{document}

\begin{frame}
\titlepage
\end{frame}

\begin{frame}
\frametitle{Introducción}
Este código es una implementación de un motor de búsqueda simple utilizando el modelo TFIDF (Term Frequency-Inverse Document Frequency) para calcular la relevancia de los documentos en función de una consulta de búsqueda.
\end{frame}

\begin{frame}
\frametitle{Funciones}
\begin{itemize}
\item LeerArchivosDeTexto()
\item Normalizar(string documento)
\item ObtenerPalabras(string nombreArchivo)
\item CalcularVectorTfIdf(string[] palabras, Dictionary<string, List<string>> frecuencia, string[] array)
\item CalcularSimilitudCoseno(double[] vector1, double[] vector2)
\item ObtenerPalabrasQuery(string query)
\item CalcularMatrizTfIdf(string[] contenido, string[] nombres)
\item CalcularSimilitudQuery(string query, string[] nombres, string[] contenido, double[,] matriz, Dictionary<string, List<string>> frecuencia)
\item Busqueda(string query)
\end{itemize}
\end{frame}

\begin{frame}
\frametitle{Funciones (cont.)}
\begin{itemize}
\item LeerArchivosDeTexto(): lee todos los archivos de texto en una carpeta llamada "Content" y devuelve una tupla con el contenido de los archivos y sus nombres.
\item Normalizar(string documento): toma un documento y lo normaliza, convirtiendo todas las palabras en minúsculas y eliminando los caracteres no alfanuméricos.
\item ObtenerPalabras(string nombreArchivo): lee el contenido de un archivo y lo normaliza, luego devuelve un array con todas las palabras del archivo.
\end{itemize}
\end{frame}

\begin{frame}
\frametitle{Funciones (cont.)}
\begin{itemize}
\item CalcularVectorTfIdf(string[] palabras, Dictionary<string, List<string>> frecuencia, string[] array): toma una lista de palabras, una frecuencia de documentos y un array de palabras, y calcula el vector TF-IDF para cada palabra en la lista.
\item CalcularSimilitudCoseno(double[] vector1, double[] vector2): calcula la similitud coseno entre dos vectores.
\item ObtenerPalabrasQuery(string query): normaliza una consulta de búsqueda y devuelve un array con todas las palabras.
\end{itemize}
\end{frame}

\begin{frame}
\frametitle{Funciones (cont.)}
\begin{itemize}
\item CalcularMatrizTfIdf(string[] contenido, string[] nombres): calcula la matriz TF-IDF para todos los documentos en una lista de contenido y nombres de archivo.
\item CalcularSimilitudQuery(string query, string[] nombres, string[] contenido, double[,] matriz, Dictionary<string, List<string>> frecuencia): calcula la similitud coseno entre una consulta de búsqueda y todos los documentos en una lista de contenido y nombres de archivo.
\item Busqueda(string query): es la función principal que realiza una búsqueda de texto. Utiliza las funciones anteriores para calcular la relevancia de los documentos en función de una consulta de búsqueda y devuelve un array de SearchItem, que contiene el nombre del archivo, la similitud coseno, las palabras clave (las palabras que aparecen en el archivo) y las primeras diez palabras del archivo.
\end{itemize}
\end{frame}

\end{document}